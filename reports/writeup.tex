\documentclass[8pt, oneside, letterpaper]{book}
\usepackage[utf8]{inputenc}
\usepackage[a4paper, total={6in, 8in}]{geometry}

\usepackage{color}   %May be necessary if you want to color links
\usepackage{hyperref}
\usepackage{blindtext}
\usepackage{enumerate}
%\usepackage{bbold}
\usepackage{scrextend}
\usepackage{amsmath}
\usepackage{amssymb}
\usepackage{xfrac}
\usepackage{bm}
\usepackage{tikz}
\usepackage{centernot}
\usepackage{amsthm}
\usepackage{framed}
\usepackage{graphicx}
\usepackage{listings}
\newtheorem{theorem}{Theorem}[section]
\newtheorem*{property}{Property}
\newtheorem{lemma}{Lemma}[section]
\newtheorem*{corollary}{Corollary}
\newtheorem*{remark}{Remark}
\newtheorem{definition}{Definition}[section]
\usepackage[tworuled,noend]{algorithm2e}

\newtheorem*{claim}{Claim}
\newtheorem*{note}{Note}

\newcommand*{\norm}[1]{\left|\left|#1\right|\right|}
\newcommand*{\Var}{\text{\normalfont Var}}
\newcommand*{\var}{\text{\normalfont Var}}
\newcommand*{\median}{\text{\normalfont median}}
\renewcommand*{\th}{^\text{th}}
\newcommand*{\R}{\mathbb{R}}
\newcommand*{\N}{\mathbb{N}}
\newcommand*{\I}{\mathbb{I}}
\renewcommand*{\[}{[\![}
  \renewcommand*{\]}{]\!]}
\newcommand{\niton}{\not\owns}
\newcommand{\rpm}{\raisebox{.2ex}{$\scriptstyle\pm$}}


\title{GrAsS - Writeup}
%\author{Louis Coulon, Joachim Desroches, Olivier Lemer}
\date{}
\begin{document}
\maketitle
%\newpage

%\paragraph{Statement}
%
%\textit{You will prepare a short writeup (max 2 pages) on the reported
%  vulnerabilities and your architecture, detailing how you mitigate different
%  attack vectors; deadline for the report is May 26 (the writeup should contain
%  the summary of the presentation as well as the following discussion)}
%
%\textit{A summary of design, implementation, and special optimizations.\\
%Discuss the discovered vulnerabilities.}
%
%\begin{itemize}
%\item Architecture, design choices, implementation, optimizations
%\item How attack vectors are mitigated
%\item Reported vulnerabilities,
%\end{itemize}
%
%\section {Introduction}


\section* {Implementation}
As a general rule, we tried to keep everything as simple as possible, as a way
to reduce the attack surface and to maintain good control over our code's
behavior. This motivated our choice of C rather than C++: we get explicit
control and responsibility of the memory, and all higher-level security-related
measures, such as sanitization or path management must be designed and
implemented by ourselves, which gives us explicit and finer-grained control.

The project's structure follows the same rule and remains simple and efficient:
our very simple server and client pair shares a common library for socket
management and file transfer.

\paragraph {Server/Client architecture}
The server is responsible for all validation and sanitization, as the client
should not -- and cannot -- be trusted: because the attacker has direct access
only to the client, this reduces the effective attack surface of the overall
system, but more importantly the server cannot rely on which client program will
be used, and an attacker could very well implement their own, malicious client.

\paragraph {Connection threads and concurrency}
Upon receiving a new connection, the server will create a new thread to manage
communication with the new client. Because the only information that is shared
between clients -- the list of users -- is immutable once the server starts
accepting connections, there is no risk of concurrency-based attacks by
well-synchronized, malicious clients. Although the actual base directory is also
shared, any concurrency risks are taken care of by the system on which the
server is run, which we assume can be trusted in that sense. Finally, all
temporary files created during command executions are unique for each user.

\paragraph {Command parsing and execution}
As much as possible, the server avoids using the shell, as this requires more
careful handling like sanitization and string concatenations, which open the
door to potential string format or buffer overflow vulnerabilities. We rather
make use of calls to functions such as \texttt{mkdirat(2)} when possible.

Similarly, the user's current working directory is explicitely maintained by the
server in the form of an absolute, canonic path. A \texttt{canonify\_path}
function is used on all manipulated paths, so that their uniform and consistent
representation makes persmission checks straightforward, while also eliminating
any tricks to get out of the base directory: either a path can be canonified and
is correct, or it is rejected.

For commands such as \texttt{grep} or \texttt{ls} for which the use of the
standard \texttt{system} function was unavoidable, sanitization of the user's
input has to be made. A sanitizer is given a set of characters to escape, and
the entire input is surrounded by quotes. All buffers -- which are
systematically null-terminated -- are allocated large enough to take into
account the size of the user's input, and its potential growth due to
sanitization, in order to avoid buffer overflows (except when we purposefully
forgot; see bug ???).

\paragraph {File transfer}
A lot of modularization was made for file transfer, since both the client and
the server execute the same code when sending or receiving a file. Both have a
function responsible for starting a new thread and a new connection for the file
transfer, which take as argument the function that must be called upon
successful connection (i.e. whether to send or receive a file), so that the
transfer thread creation code can be reused for both the \texttt{put} and the
\texttt{get} command. The part of the code that is not shared between client and
server is the one responsible for creating a new socket and managing the
connection, as it is always the server that will open a new connection and
listen on it.

\section* {Vulnerabilities}
Among the following vulnerabilities that we hid in our code, only the code
injection has been discovered and reported by 5 other teams. We here explain how
they work and how to fix them.

\paragraph {Vulnerability 1: Stack Buffer Overflow}

This exploit takes advantage of the fact that sanitization of the user input can
double the size of the string, as any character can potentially be replaced
itself and a backslash preceeding it.

When calling the sanitizer, a buffer is allocated on the stack, and the
sanitization function is tasked to fill it with a sanitized version of the
user's input. Because this receiving buffer has the same size as the
non-sanitized buffer, sanitization might overflow it and allow a
buffer-overflow. Allocating a buffer twice the current size would fix this
vulnerability.

\paragraph {Vulnerability 2: Stack Buffer Overflow}
Upon receiving the server's answer to a \texttt{put} command, the client creates
a thread that will send the file to the server. The buffer the client uses to
read from the file has max size \texttt{MAX\_LINE\_SIZE}, while the server uses
a buffer of size \texttt{MAX\_TR\_SIZE}, which is set to be smaller.
Additionally, the server does something stupid by computing the minimum between
the expected received length, and the maximum transfer size. However the
\texttt{min} macro is wrong:
\begin{lstlisting}[language=C]
  #define min(a, b) ((size_t)(int)((ssize_t)(a) < (b) ? (a) : (b)))
\end{lstlisting}
We can therefore choose a value for \texttt{file\_len} such that it is smaller
than \texttt{MAX\_TR\_SIZE} if cast as a \texttt{ssize\_t}, but \emph{larger} if
cast as an \texttt{int}. Such a value can be, for example, $2^{8 * 8 - 1}$,
which casts to a negative \texttt{ssize\_t}, and to zero as an \texttt{int}.
Since the receiving buffer has size \texttt{MAX\_TR\_SIZE}, this allows a buffer
overflow. A correct implementation of the \texttt{min} macro would solve this
issue.

\paragraph {Vulnerability 3: Format String Vulnerability}
When receiving a file, the server uses \texttt{fprintf(file, recv\_buffer)}. The
content of the file can thus be used to take advantage the format string
vulnerability: reading the receiving file can be used for inspection in order to
ultimately overwrite a return address to gain flow control by sending a file
containing a string of the form
\texttt{\%<hijack\_16lsb>c\%<buffer + ptr\_offset>\$hn<ra\_location>}. Replacing
the line with \texttt{fprintf(file, ``\%s'', recv\_buffer)} would solve this issue.

\paragraph {Vulnerability 4: Command Injection Vulnerability}
As presented above, user-provided strings are usually sanitized. However it is
done only when directly fed to the \texttt{system} function; the argument to
\texttt{mkdirat} is not sanitized. This allows an attacker to inject a command
into a folder name, cd into it and execute a command that will use the current
directory's absolute path, such as \texttt{ls} or \texttt{grep}. This can be
fixed by sanitizing \emph{all} user input.

\paragraph {Vulnerability 5: Buffer Overflow}
Upon executing \texttt{cd}, the server will concatenate the provided relative
path to the current working directory, before trying to canonify it. It will
expect the resulting concatenation to be twice the size of a maximum path, which
is actually less than the accepted max size of a user's input. Therefore the
concatenated string might overflow its buffer.



\end{document}
